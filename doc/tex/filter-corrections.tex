\section{Filter corrections}
    
\noindent In Section \ref{sec:sampling} we have discussed the fact that the
$\Wlow$ filter is exact for sampling, in the sense that preserves all Fourier
modes up to $k_N$ and kills all other modes above, but it is computationally
expensive to apply. Hence we use other filters with a compact support to reduce
the computational complexity of a sample to ${\cal O}(1)$.
Usually these alternative filters, let's call them $W$,
are symmetric about $x=0$,
and if we try to insert it into equation (\ref{eq:sample1}) 
we must guarantee that $W$ have been defined in the domain $[-L,L]$.
On the other hand, the Fourier
expansion in equation (\ref{eq:sample2}) is only possible if
$W$ is $L$-periodic. 
Hence $W$ must satisfy:
$$W(L-\epsilon)=W(-\epsilon) = W(\epsilon).$$
That in fact is indeed true for $\Wlow$, but if we introduce
it by hand if we use another filter
otherwise we might get unexpected results.

Since these filters $W$ have non uniform modes below $k_N$
their convolution with $\rho_o$ does not preserve $\langle u_k, \rho_o
\rangle$. But we could use (\ref{eq:sample2}) to recover 
$\langle u_k, \rho_o \rangle$ from $\langle u_k, W * \rho_o \rangle$.
However, recall that we do not compute $\langle u_k, \rho \rangle$
directly but we use $\tilde \rho_k$ as proxy:
\begin{equation}
    \tilde \rho_k
    = \sum_{z\in\Z} 
        \langle u_{k + z\, N}, \rho \rangle
    = \sum_{z\in\Z} 
       L 
       \langle u_{k + z\, N}, W \rangle 
       \langle u_{k + z\, N}, \rho_o \rangle .
\end{equation}
If $\langle u_{k}, W \rangle $ is assumed to be leading
in order with respect to all the rest of the aliases $k+z,N$,
then we can approximate the previous equation and we obtain
\begin{equation}
    \langle u_{k}, \rho_o \rangle 
    \approx  
    \frac{  
        \tilde \rho_k %
    }{
        L\, %
       \tilde W_k
    },
    \label{eq:correction1}
\end{equation}
which should be a better estimator for 
    $\langle u_{k}, \rho_o \rangle $ than $\tilde \rho_k$ alone.
Equation (\ref{eq:correction1}) is what we call the \emph{sampling filter
correction}.



{\centering%
\begin{table}
    \begin{tabular}{cc}
        \begin{subfigure}{.5\textwidth}
            \centering\includegraphics[width=\columnwidth]{./plots/NGP-modes-20.pdf}
        \end{subfigure} &
        \begin{subfigure}{.5\textwidth}
            \centering\includegraphics[width=\columnwidth]{./plots/CIC-modes-20.pdf}
        \end{subfigure} \\
        \begin{subfigure}{.5\textwidth}
            \centering\includegraphics[width=\columnwidth]{./plots/PCS-modes-20.pdf}
        \end{subfigure} &
        \begin{subfigure}{.5\textwidth}
            \centering\includegraphics[width=\columnwidth]{./plots/TSC-modes-20.pdf}
        \end{subfigure} \\
        \begin{subfigure}{.5\textwidth}
            \centering\includegraphics[width=\columnwidth]{./plots/gaussian-modes-20.pdf}
        \end{subfigure} &
        \begin{subfigure}{.5\textwidth}
            \centering\includegraphics[width=\columnwidth]{./plots/low_pass-modes-20.pdf}
        \end{subfigure} \\
    \end{tabular}
    \caption{Different sampling kernels at $N=20$, with a boxsize $L=1$.
    The Fourier modes are compared to the discrete Fourier transform modes.
    }
    \label{tab:wfilters}
\end{table}
}
