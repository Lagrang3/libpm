\section{Sampling filters}

\noindent In general, the density field will not satisfy 
the condition necessary for Theorem \ref{th:nyquist} to hold,
that's because $\rho_o$, see equation (\ref{eq:rhoo}) is
made up of the sum of an infinite number of Fourier modes.
That in practice provokes the fact that when we sample
$\rho_o$ at the grid points we get zero everywhere unless
there is one of the particles lying precisely on top of grid
point, due to the Dirac deltas.

One possible solution to this \emph{sampling problem}
would be to apply a low-pass filter to $\rho_o$
thus discarding all modes above $k_N$.
And the error we would introduce would be the approximation of
$\rho = \Wlow * \rho_o$ to $\rho_o$. 
With
\begin{equation}
    \rho(x) = \int dx' \Wlow(x-x') \rho_o(x'),
\end{equation}

The low-pass filter can actually be computed analytically
and its convolution with Dirac deltas would be again itself.
Considering the Fourier series for $\Wlow$ we get 
\begin{align}
\begin{split}
    \langle u_k, \rho\rangle
    &= \int dx' \sum_{k'} \langle u_{k'},\Wlow \rangle \delta_{kk'}   u_{k'}^{\dagger}(x') \rho_o(x') \\
    &= L \langle u_{k},\Wlow \rangle \langle u_{k}, \rho_o \rangle ,
\end{split}
\end{align}
therefore
\begin{equation}
\langle u_{k},\Wlow \rangle = \begin{cases}
    0, & |k|>k_N\\
    1/L, & |k|\le k_N
    \end{cases}
\end{equation}
and $\Wlow$ can be reconstructed analytically from its Fourier expansion:
\begin{equation}
 \Wlow(x)
 = \frac{1}{L}\sum_{k=-n_N}^{k_N} u_k(x)
 = \frac{1}{L} \frac{  \sin(\frac{\pi}{L} x (2k_N +1))  }{\sin(\frac{\pi}{L}x)}
 \label{eq:wlow}
\end{equation}
From equation (\ref{eq:wlow}) we see that $\Wlow$ is a distribution
which wiggles and decays slowly as we go further from the origin,
see figure \ref{fig:wdelta}.
\begin{figure}
    \centering\includegraphics[width=.7\hsize]{./plots/delta-W.pdf}
    \caption{The low-pass filter $\Wlow$ for $N=64$ and $L=1$.}
    \label{fig:wdelta}
\end{figure}
Having as consequence that in order to sample the density field $\rho_o$
we would need to add the contribution
of every particle at every grid point, leading to an 
algorithmic complexity, for a 1-dimesional grid, of $\mathcal{O}(N_p \cdot N)$, which is not feasible
since usually one chooses $N$ to be as big as the memory permits and that's roughly $N\approx N_p$.

To lower the complexity to $\mathcal{O}(N_p)$ one needs to approximate 
the density field produced by a point particle to a distribution with a compact support
such as: nearest-grid-point (NGP), cloud-in-cell (CIC), triangular-shaped-cloud (TSC)
or a piecewise-cubic-spline (PCS), see \cite{sefusatti} and \cite{hockney}.
\begin{equation}
    W^{\text{NGP}}(s L/N) = 
    \frac{N}{L} \times
    \begin{cases} 
        1, & |s|< 1/2 \\
        0, &\text{otherwise}
    \end{cases}
\end{equation}
\begin{equation}
    W^{\text{CIC}}(s L/N) = 
    \frac{N}{L} \times
    \begin{cases} 
        1-|s|, & |s|< 1 \\
        0, &\text{otherwise}
    \end{cases}
\end{equation}
\begin{equation}
    W^{\text{TSC}}(s L/N) = 
    \frac{N}{L} \times
    \begin{cases} 
        3/4 - s^2, & |s|< 1/2 \\
        \frac{1}{2}(3/2 - |s|)^2, & 1/2 \le |s|< 3/2 \\
        0, &\text{otherwise}
    \end{cases}
\end{equation}
\begin{equation}
    W^{\text{PCS}}(s L/N) = 
    \frac{N}{L} \times
    \begin{cases} 
        \frac{1}{6}(4-6s^2+3|s|^3), & |s|< 1 \\
        \frac{1}{6}(2-|s|)^3, & 1\le |s| <2 \\
        0, &\text{otherwise}
    \end{cases}
\end{equation}
in all cases the prefactor ensures that $\int_{-L/2}^{L/2} dx W(x)  =1$.

The only objection I have to those filters is the presence of non-negligible
modes higher than $k_N$. See for instance in
the Table \ref{tab:wfilters} what happens if we apply $\Wlow$ to
NGP, CIC, TSC, PCS and finally to a gaussian shape with $\sigma = L/N$.
See also the power spectrum of all proposed filters in figure \ref{fig:pk}.
That means two things:
\begin{enumerate}
    \item The filters NGP, CIC, TSC and PCS though they might be good
    to reduce the non-locality of a density function for every particle,
    they will introduce aliasing effects when sampled because
    they present non-negligible mode amplitudes for $|k|>k_N$,
    \item the gaussian filter doesn't have a compact support
    but at $k_N$ and beyond the modes are suppressed and 
    the condition for theorem \ref{th:nyquist} is approximately satisfied,
    so there will be a very small aliasing effect.
\end{enumerate}

{\centering%
\begin{table}
    \begin{tabular}{cc}
        \begin{subfigure}{.5\textwidth}
            \centering\includegraphics[width=\columnwidth]{./plots/NGP-W.pdf}
        \end{subfigure} &
        \begin{subfigure}{.5\textwidth}
            \centering\includegraphics[width=\columnwidth]{./plots/CIC-W.pdf}
        \end{subfigure} \\
        \begin{subfigure}{.5\textwidth}
            \centering\includegraphics[width=\columnwidth]{./plots/TSC-W.pdf}
        \end{subfigure} &
        \begin{subfigure}{.5\textwidth}
            \centering\includegraphics[width=\columnwidth]{./plots/PCS-W.pdf}
        \end{subfigure} \\
        \begin{subfigure}{.5\textwidth}
            \centering\includegraphics[width=\columnwidth]{./plots/gaussian-W.pdf}
        \end{subfigure} &
        \begin{subfigure}{.5\textwidth}
        \end{subfigure} \\
    \end{tabular}
    \caption{Different sampling kernels at $N=64$, with a boxsize $L=1$.
    The shape of the kernel and its convolution with a low-pass filter
    are compared. The gaussian shape changes only slightly.}
    \label{tab:wfilters}
\end{table}
}
\begin{figure}
    \centering\includegraphics[width=.7\hsize]{./plots/pk.pdf}
    \caption{Different sampling kernels at $N=64$. The gaussian
    modes decay exponentially.}
    \label{fig:pk}
\end{figure}


In fact for the gaussian filter
\begin{equation}
    W^{\text{G}}(x) = 
    \frac{1}{\sqrt{2\pi \sigma^2}} \exp\left(-\frac{x^2}{2\sigma^2}\right),
\end{equation}
we can compute exactly the modes amplitude
\begin{equation}
    \langle u_k, W^{\text{G}} \rangle = 
    \frac{1}{L} \exp\left(-\frac{k^2}{2\eta^2}\right),
\end{equation}
where $\sigma \eta = \frac{L}{2\pi}$.
Then the bigger $\sigma$ the better the approximation 
$\Wlow * W^G = W^G$, since $\eta$ becomes smaller,
however there will be an increase in the non-locality 
of a single particle density field and thus the computational
cost of sampling.
To ilustrate this point we can parametrize $\sigma = \alpha \frac{L}{N}$
and consequently we get $\eta = \frac{N}{2\pi \alpha}$.
Then the amplitude of the $k_N$ mode with relative to $k=0$
becomes:
\begin{equation}
    \frac{\langle u_{k_N}, W^G \rangle}{ \langle u_0, W^G \rangle}
    \approx \exp\left( -\frac{\pi^2 \alpha^2}{2} \right),
\end{equation}
while the amplitude of $W^G$ at grid point $x_n$ relative to $x_0$
becomes
\begin{equation}
    \frac{W^G(x_n)}{W^G(0)}
    = \exp\left( -\frac{n^2}{ 2 \alpha^2} \right),
\end{equation}
For $\alpha=1$ we have
\begin{align}
\begin{split}
    \frac{\langle u_{k_N}, W^G \rangle}{ \langle u_0, W^G \rangle} 
        &\approx 7\times 10^{-3}, \\
    \frac{W^G(x_4)}{W^G(0)} 
        &\approx 3\times 10^{-4},
\end{split}
\end{align}
So I guess Theorem \ref{th:nyquist} can be used safe from aliasing
and we can sample for every particle its gaussian density field 
at $3\times 2 =6$ nearest grid points without problems.
In fact if we apply a cut-off to $W^G$ for $n > 4$ we get the
spectrum show in figure \ref{fig:pkgausscut} and
its convolution with $\Wlow$ in figure \ref{fig:wlowgausscut}.

\begin{figure}
    \centering\includegraphics[width=.7\hsize]{./plots/pk-gauss-cut.pdf}
    \caption{Power spectrum ($N=64$) 
    of the gaussian filter with cut-off for $n>4$.}
    \label{fig:pkgausscut}
\end{figure}
\begin{figure}
    \centering\includegraphics[width=.7\hsize]{./plots/wlow-gauss-cut.pdf}
    \caption{Convolution of the gaussian with cut-off for $n>4$ 
    with a low-pass filter ($N=64$).}
    \label{fig:wlowgausscut}
\end{figure}
