\section{Code units}

\noindent Numbers are represented in the computers with no reference
to units of measure.
Therefore computer programmers are more than happy to work
with adimentional quantities.
However this is not always possible and the freedom 
of physical dimensions of length, time and mass,
introduces aditional parameters to the code contributing to 
possible sources of mistakes.
For the problem we are dealing here the natural choice of
units is the following:
\begin{itemize}
    \item length: use $L$ the size of the box,
    \item time: the Hubble time $H_o^{-1}$,
    \item mass: is not needed (for the moment).
\end{itemize}

With this definition we see that the factor $Gm$,
which are needed to compute the interaction between particles
is determined by the number of particles $N_p$ and the 
matter density $\Omega_{m}$, because
at $z=0$ we have:
\begin{equation}
    \Omega_m = \rho_m/\rho_c = \frac{ N_p m }{ L^3} \frac{8\pi G}{ 3 H_o^2},
\end{equation}
therefore
\begin{equation}
    Gm = \frac{ 3 \Omega_m }{ 8\pi N_p} [L^3 H_o^2],
\end{equation}
for the computer the terms inside the square bracket can be omitted.

To produce consistent computations, the input values: 
$H_o$, $L$, $\vec x_i$ and $\vec u_i$ must be expressed using the same
system of units, that's the only requirement.
It doesn't matter if it is SI or CGS or any other combination,
for instance Mpc/h for length and year/h for time.
The output of the program will be given in the same units
as the input.

For example we use the following formulas to translate
the input numbers into the internal representation:
\begin{align}\
    \begin{split}
    \vec x_i &= \frac{ \vec x_i }{ L} [L]\\
    \vec u_i &= \frac{ \vec u_i }{ L H_o} [L H_o]\\
    \end{split}
\end{align}

\textsc{Gevolution}'s and \textsc{Gadget}'s default units for snapshots 
are $10^{10}M_{\odot}/h$ for masses, $\text{kpc}/h$ for lengths
and $\text{km/s}$ for velocities. 
The ratio between the unit of length and velocity
$\frac{\text{kpc}/h}{\text{km/s}} = h^{-1}\,\text{s}\,\text{kpc}/\text{km}$ 
produces the intrisic unit of time for those snapshots.
To ensure those snapshots are read correctly (the velocities) one must specify 
the length of the box $L$ in units of $\text{kpc}/h$ and the Hubble parameter
in units of $h\,\text{km}\,\text{s}^{-1}\,\text{kpc}^{-1}$,
which is actually a trivial thing because 
$H_o = 0.1 [h\,\text{km}\,\text{s}^{-1}\,\text{kpc}^{-1}]$ exactly
due to the definition of $h$.


