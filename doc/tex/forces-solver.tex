\section{Force solver}

\noindent The core of a pure PM code is constituted
by the routines that compute the forces acting on the particles
once their positions are fed as input.
The \emph{Mesh} is a 3-dimensional grid of length $N$,
that is used to represent the continuous density $\rho$,
potential $\phi$ and force $\vec F$ fields.
The steps for computation can be summarized as follows (see \cite{hockney}):
\begin{enumerate}
    \item assign the charges to the mesh, ie. compute $\rho$
    from the knowledge of $\{ \vec x_i \}_{i=1\ldots N_p}$,
    \item solve the Poisson equation to compute $\phi$
    out of $\rho$,
    \item compute the force field $\vec F$ 
    from the potential $\phi$,
    \item interpolate $\vec F$ from the mesh to the particles'
    positions.
\end{enumerate}
In this experiment we will deal with one particular
PM method that solves the differential equations that
relate the different fields in the Fourier space.

Let's for simplicity start the discussion
in one dimension.
The grid then helps to represent fields in 
the cubic box $[0,L]$,
and we will assume they are two times differentiable
and square integrable.
Those fields form a Hilbert space called Sobolev,
here we will denote that space as $\H$,
with inner product $\langle,\rangle$ such that:
\begin{equation}
    \langle A,B \rangle = \frac{1 }{ L} \int_{0}^L dx A^{\dagger}(x) B(x).
\end{equation}
Then according to Fourier's theorem the set $\{ u_n \}_{n\in \Z}$,
where $u_n(x) = \exp(i\frac{2 \pi}{ L} x n)$, is an orthonormal
basis of $\H$.
That is, for any $f \in\H$ we can write:
\begin{equation}
    f(x) = \sum_{k=-\infty}^{\infty}\, \langle u_k, f \rangle u_k(x).
\end{equation}
Notice that $f$ is real-valued function if and only if
$\langle u_k, f\rangle^* = \langle u_{-k},f \rangle$.

My very first idea to this method was to take advantage
of Nyquist--Shannon sampling theorem (see \cite{nyquist}).
Starting from a real density field $\rho_o$, ie. the one
computed for point particles:
\begin{equation}
    \rho_o(\vec x) = \sum_{i} G m\, \delta(\vec x - \vec x_i),
    \label{eq:rhoo}
\end{equation}
use a \emph{low-pass} filter $\Wlow$ to eliminate frequency
modes higher than Nyquist's $k_{\mathrm{nyquist}} \approx N/2$:
\begin{equation}
    \rho(\vec x) = (\Wlow*\rho_o)(\vec x),
\end{equation}
where $*$ denotes the convolution operator:
\begin{equation}
    (A*B)(x) = \int_0^L dx' A(x-x') B(x').
\end{equation}

The approximation we are making by using the mesh instead of
a continuous field is all encoded in the difference 
$\rho-\rho_o$, because
Nyquist's theorem guarantees that $\rho$ can
be recovered exactly by the only knowledge
of the finite grid points. 
Further on $\phi$ and $\vec F$ can be computed exactly
in Fourier space and brought back to position space.
The interpolation of $\vec F$ to particles position
can again be computed exactly based on the sampling theorem.

