\section{Force interpolation}

\noindent Forces are going to be computed as
with finite differences of the potential $\phi$.
It is the value of $\phi$ what we need to interpolate 
from the knowledge of the value it takes on 
the grid points.

Once $\tilde\phi$ is known for all modes on the grid,
then according to Theorem \ref{th:nyquist} 
we are able to recover $\phi(x)$ exactly 
at every point $0 \le 0 < L$ using the Fourier expansion:
\begin{equation}
    \phi(x) = \sum_{k=-k_N}^{k_N} \tilde \phi_k\, u_k(x).
\end{equation}
However, using the previous equation would imply a computational
cost of $\mathcal{O}(N)$ for the computation of the force
acting on every particle. Too expensive to compute in practice.
A method involving $\mathrm{O}(1)$ operation must be deduced
for that task, since the discrete Fourier transform 
of $\tilde\phi_k$ gives the values $\phi_n = \phi(x_n)$
of the potential at the grid points.
And for the points outside there must an interpolation
formula that computes the exact or approximate value
of $\phi$ from the knowledge of $\phi_n$ at neighboring grid points.
Different degrees of splines can be constructed for this
purpose.

The reconstruction of the function $\phi$ everywhere
from the values $\phi_n$ can be made explict.
Let's us start by evaluating the convolution $\phi * W$,
where $W$ is not yet specified:
\begin{align}
    \begin{split}
        \left( \phi* W \right)(x) 
            &= \int dx'\, \phi(x-x')\, W(x') \\
            &= \sum_{k=-\infty}^{\infty}\int dx'\,\langle u_k,\phi\rangle\,u_k(x-x')\, W(x')\\
            &= \sum_{k=-\infty}^{\infty}\int dx'\,\langle u_k,\phi\rangle\,
                u_k(x)\,u_k^{\dagger}(x')\, W(x') \\
            &= \sum_{k=-\infty}^{\infty} \,\langle u_k,\phi\rangle\,
                u_k(x)\, L\langle u_k, W\rangle. \\
    \end{split}
\end{align}
If $\phi$ satisfies the necessary conditions
for Theorem \ref{th:nyquist}, then 
we write $\langle u_k,\phi\rangle$ as a sum over the sampled values
at grid points
\begin{align}
    \begin{split}
        \left( \phi* W \right)(x) 
            &=\frac{1}{N} \sum_{n=0}^{N-1} \sum_{k=-\infty}^{\infty} \,
                \phi_n\, \omega^{-kn} \,
                u_k(x)\, L\langle u_k, W\rangle \\
            &=\frac{1}{N} \sum_{n=0}^{N-1} \sum_{k=-\infty}^{\infty} \,
                \phi_n\,
                u_k\left(x-\frac{n}{N}L\right)\, L\langle u_k, W\rangle \\
            &=\frac{L}{N} \sum_{n=0}^{N-1} \,
                \phi_n\,
                 W\left(x-\frac{n}{N}L\right). \\
    \end{split}
\end{align}
If it happens that $\phi = \phi* W$, like in the case of $\Wlow$ we
obtain:
\begin{equation}
    \phi(x) = 
            \frac{L}{N} \sum_{n=0}^{N-1} \,
                \phi_n\,
                 \Wlow\left(x-\frac{n}{N}L\right).
   \label{eq:whittaker}
\end{equation}
And that's an exact result knows as \emph{Whittaker-Shannon interpolation formula}.
Unfortunately, equation (\ref{eq:whittaker}) gives us again 
an algorithmic complexity of $\mathcal{O}(N)$ for the interpolation
since $\Wlow$ is non-local.
The literature \cite{hockney} proposes the use of the formula (\ref{eq:whittaker})
with other filters which have a compact support, like NGP, CIC, etc,
but bear in mind that we commit to an error than can be traced
back to the fact that $\phi \ne \phi * W$.

\subsection{\bf $N$ odd vs even debate}

\noindent Notice for $N$ odd we can choose $\Wlow$ with cut-off frequency that satisfies 
$2 k_N + 1 = N$. Then $\Wlow$ takes a form that we call \emph{Sinc} filter
because of its resemblance and similar role to the Sinc function:
\begin{equation}
    \Wsinc(x)
     = \left.\Wlow\right|_{N\text{odd}}(x)
  = \frac{1}{L} \frac{  \sin(\frac{\pi N}{L} x)  }{\sin(\frac{\pi}{L}x)}.
  \label{eq:sinc}
\end{equation}
Interesting is the fact that $\Wsinc$ for any $N$ will satisfy equation
(\ref{eq:whittaker}) for any $x = k L / N$, ie.
\begin{equation}
    \phi_k = 
            \frac{L}{N} \sum_{n=0}^{N-1} \,
                \phi_n\,
                 \Wsinc\left(\frac{(k-n)}{N}L\right),
    \label{eq:grid_pts_identity}
\end{equation}
simply because
\begin{equation}
    \Wsinc( k L/N)
  = \frac{1}{L} \frac{  \sin(\pi k)  }{\sin(\pi k/N)}.
  = \frac{1}{L} \times 
    \begin{cases}
    N, & \text{$N$ divides $k$ }\\
    0, & \text{otherwise}
    \end{cases}
\end{equation}
and that's independent of the values of $\phi(x)$ at the grid points.
That tempt us to use $\Wsinc$ instead of $\Wlow$ as
a reference \emph{exact} interpolator.

Trouble comes when $N$ is even.
In that case $k_N = ( N-2 )/ 2$ according to Theorem \ref{th:nyquist}.
A signal with mode $k_N+1$ cannot be completly reconstructed from 
the $N$ samples, hence $\Wlow$ is has a frequency cut-off at $k_N$,
having $2 k_N+1$ modes, one mode short from $N$.
That has consequences, the first is that the identity 
(\ref{eq:grid_pts_identity}) is not satisfied by $\Wlow$
unless $\phi = \Wlow * \phi$.
On the other hand, $\Wsinc$ does satisfy (\ref{eq:grid_pts_identity})
but it can be shown that it does not interpolate a function $\phi$
with modes below $k_N$. The reason lies in the fact that
the modes of $\Wsinc$ below $k_N$ are not uniform, see figure
\ref{fig:sinc_power}.

\begin{figure}
   \begin{subfigure}[b]{0.5\textwidth}
         \centering
         \includegraphics[width=\textwidth]{plots/Sinc-modes-20.pdf}
     \end{subfigure}
   \begin{subfigure}[b]{0.5\textwidth}
         \centering
         \includegraphics[width=\textwidth]{plots/Sinc-modes-21.pdf}
     \end{subfigure}
        \caption{$\Wsinc$ power spectrum, discrete FT and true FT.
        For $N$ odd, we see the same features of $\Wlow$ that makes
        $\Wsinc$ an exact interpolator.
        While for $N$ even, the modes below $k_N$ are not flat
        and that makes $\Wsinc$ fail to be an exact interpolator.}
        \label{fig:sinc_power}
\end{figure}

\begin{figure}
   \begin{subfigure}[b]{0.5\textwidth}
         \centering
         \includegraphics[width=\textwidth]{plots/Sinc-interpolation-20.pdf}
     \end{subfigure}
   \begin{subfigure}[b]{0.5\textwidth}
         \centering
         \includegraphics[width=\textwidth]{plots/Sinc-interpolation-21.pdf}
     \end{subfigure}
        \label{fig:sinc_interpolation}
\end{figure}

\begin{figure}
   \begin{subfigure}[b]{0.5\textwidth}
         \centering
         \includegraphics[width=\textwidth]{plots/low_pass-interpolation-20.pdf}
     \end{subfigure}
   \begin{subfigure}[b]{0.5\textwidth}
         \centering
         \includegraphics[width=\textwidth]{plots/low_pass-interpolation-21.pdf}
     \end{subfigure}
        \label{fig:sinc_interpolation}
\end{figure}
