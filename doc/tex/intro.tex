\section{Introduction}
    
    \subsection{Why a library?}
    \noindent Take for instance \texttt{sqrt}.
    It is not a program, it is an abstract concept---not a function
    because a function would be a specific implementation of the
    root square---that you call from your program.
    It is useful because it performs a common operation
    (that makes everyone need it) and
    it is non-trivial to implement.
    You don't do cosmological simulations with \texttt{sqrt}
    alone, but you definitely needed.
    
    A library---just like \texttt{sqrt}---does a well defined task,
    leaving out space to fill by the users, in order
    to increase its range of applications.
    It is mantained by the authors.
    And the users get improvements for free, ie. 
    faster execution, higher precision, debugging,
    and extended features. For instance \texttt{sqrt<complex>}.
    Furthermore, a library can outlive the codes which use it.
    \texttt{Gadget} some day will not be mantained anymore,
    but the \texttt{sqrt} will still be used by the forthcoming
    cosmological codes.
